\documentclass[a4paper,12px]{article}
\usepackage{graphicx}
\usepackage[english]{babel}
\usepackage{fancyhdr}
\usepackage{lastpage}
\usepackage{xifthen}
\usepackage[linesnumberedhidden, titlenotnumbered]{algorithm2e}
\usepackage{lipsum}
\usepackage{hyperref}
\usepackage{array}
\usepackage{tabularx}
\usepackage{caption}
\usepackage{amsfonts}

\usepackage{minted}
\usepackage{listings}
\renewcommand\listingscaption{Code}

\pagestyle{fancy}
\lhead{\includegraphics[width=7cm]{logoUvA}}
\rhead{\footnotesize \textsc {Report\\ \opdracht}}
\lfoot
{%
    \footnotesize \studentA
    \ifthenelse{\isundefined{\studentB}}{}{\\ \studentB}
    \ifthenelse{\isundefined{\studentC}}{}{\\ \studentC}
    \ifthenelse{\isundefined{\studentD}}{}{\\ \studentD}
    \ifthenelse{\isundefined{\studentE}}{}{\\ \studentE}
}
\cfoot{}
\rfoot{\small \textsc {Page \thepage\ of \pageref{LastPage}}}
\renewcommand{\footrulewidth}{0.5pt}

\fancypagestyle{firststyle}
{%
    \fancyhf{}
    \renewcommand{\headrulewidth}{0pt}
    \chead{\includegraphics[width=7cm]{logoUvA}}
    \rfoot{\small \textsc {Page \thepage\ of \pageref{LastPage}}}
}

\setlength{\topmargin}{-0.3in}
\setlength{\textheight}{630pt}
\setlength{\headsep}{40pt}

% =================================== DOC INFO ===================================

\newcommand{\titel}{Algorithms and Complexity}
\newcommand{\opdracht}{Week 1: Assignments}
\newcommand{\docent}{Dr. Leen Torenvliet}
\newcommand{\cursus}{Algorithms and Complexity}
\newcommand{\vakcode}{5062ALCO6Y}
\newcommand{\datum}{\today}
\newcommand{\studentA}{Robin Klusman}
\newcommand{\uvanetidA}{10675671}
\newcommand{\studentB}{Maico Timmerman}
\newcommand{\uvanetidB}{10542590}
\newcommand{\studentC}{Boudewijn Braams}
\newcommand{\uvanetidC}{10401040}
\newcommand{\studentD}{Govert Verkes}
\newcommand{\uvanetidD}{10211748}
%\newcommand{\studentE}{Naam student 5}
\newcommand{\uvanetidE}{UvAnetID student 5}

% ===================================  ===================================

\begin{document}
\thispagestyle{firststyle}
\begin{center}
    \textsc{\Large \opdracht}\\[0.2cm]
    \rule{\linewidth}{0.5pt} \\[0.4cm]
    {\huge \bfseries \titel}
    \rule{\linewidth}{0.5pt} \\[0.2cm]
    {\large \datum  \\[0.4cm]}

    \begin{minipage}{0.4\textwidth}
        \begin{flushleft}
            \emph{Supervisor:} \\
            \docent \\[0.2cm]
            \emph{Student:}\\
            {\studentA \\ {\small \uvanetidA \\[0.2cm]}}
            \ifthenelse{\isundefined{\studentB}}{}{\studentB \\ {\small \uvanetidB \\[0.2cm]}}
        \end{flushleft}
    \end{minipage}
    ~%
    \begin{minipage}{0.4\textwidth}
        \begin{flushright}
            \emph{Course:} \\
            \cursus \\[0.2cm]
            \emph{Student:}\\
            \ifthenelse{\isundefined{\studentC}}{}{\studentC \\ {\small \uvanetidC \\[0.2cm]}}
            \ifthenelse{\isundefined{\studentD}}{}{\studentD \\ {\small \uvanetidD \\[0.2cm]}}
            \ifthenelse{\isundefined{\studentE}}{}{\studentE \\ {\small \uvanetidE \\ [0.2cm]}}
        \end{flushright}
    \end{minipage}\\[1 cm]
\end{center}


% =================================== CONTENTS ===================================

\tableofcontents
\clearpage

% =================================== MAIN TEXT ===================================

\section{Assignment 1}
\subsection{(a)}
The following code is our implementation of the algorithm we devised to generate all possible permutations
of a sequence of numbers ranging from 1 to any positive integer $n$ (that is to be provided by the user).\\
It uses recursion to find all permutations. Say we have a list with $k$ elements, then to find all permutations,
the algorithm will first swap an item of the list with the first one, and `lock' that item. Then it does the
same with the rest of the list by incrementing the $index\_start$ by one, this index keeps track of the
position that is being swapped to. As soon as the $index\_start$, and the $index\_stop$ are the
 same, there is only one possible permutation left. This will then be added to
 our list of permuations.

\begin{listing}[H]
\inputminted{python}{1a.py}
\caption{Calculating all permutations of 1,2,3 \ldots n}
\end{listing}

\subsection{(b)} Here our algorithm calculated the $n$th permuation by
determining the factors of factorial numbers. This generates a factorial code
which is then used to determine which item should be taken from the list next to
form our $n$th permutation.  In this way we found that the
5170403347776995327994765th permutation of a list from 0 to 24, which is the
same as 1 to 25 but better computable, is the following list. \{8, 7,
    24, 23, 22, 21, 20, 19, 18, 17, 16, 15, 14, 13, 12, 11, 10, 6, 9, 4, 1, 5,
    2, 3, 0\}

\begin{listing}[H]
\inputminted{python}{1b.py}
\caption{Calculating the n'th permutation of 1,2,3 \ldots n}
\end{listing}

\subsection{(c)}

If we assume all of our arithmetic operations take one step, our algorithm will
need about $n*(n^{2} + 8)$ steps to find the list we found above in 1b.

\section{Assignment 2}
\subsection{(a)}
$f(n) \in o(1)$ contains function that for a $\lim_{n \to \infty}$ will become
zero.\\
$n^{o(1)}$ are functions $\forall \epsilon \in \mathbb{R}_{>0}$ ($g
\in o(n^{\epsilon})$)

\subsection{(b)}
$$f(n) = (n^{n})^{\frac{1}{n}}$$

\section{Assignment 3}
\subsection{(a)}
\begin{algorithm}[H]
    \KwIn{A collection S of n strings over an alphabet A of k symbols}
    \KwOut{An alphabetically sorted list of the given n strings}
    \Indp
    Take the next letter (starting with the first) of your alphabet \\
    Create a subset of S with all strings beginning with the same letter \\
    \Indp
        For the created subset, move the to next letter in your string \\
        Create a subset of our subset with all strings that have the same on position defined above \\
        Repeat for every symbol in A \\
        Repeat for every position in the strings \\
    \Indm
    Repeat for each symbol in A \\
    \bigskip
    \caption{Sorting of a collection of n strings over an alphabet of k symbols}
\end{algorithm}

\subsection{(b)}
    $$T(n) =
        \left\{
            \begin{array}{ll}
                c & n < 2  \\
                k*T(\frac{n}{k}) + f(n) &  n \geq 2 \\
            \end{array}
        \right.$$

\subsection{(c)}
Since f(n) just does sorting of strings based on a single character we see that
$$f(n) \in \theta(n*log(n)).$$ \\
Using the Akra-Bazzi method we see that a = k and b = k, so klog(k) = 1 (k being the number of symbols in
our alphabet. This fits perfectly in simplification number two described in the
course syllabus on page 29.\\

$$f(n) \in \theta(n^{log_k(k)} * log^{1}(n))$$
$$f(n) \in \theta(n*log(n))$$
Because this is true, the following can be implied:
$$T(n) \in \theta( n^{log_k(k)} * log^{1+1}(n))$$ $$T(n) \in \theta(n * log^{2}(n))$$

\subsection{(d)}
Instead of k symbols we now have log(n) symbols.
$$T(n) = log(n) * (n/log(n)) + f(n)$$ \\
The complexity of f(n) stays the same: $$\theta(n*log(n))$$
And when k is substituted for log(n) the equations stays exactly the same,
because $$log_{log(n)}{log(n)} = 1$$
and so replacing k with log(n) has no influence on the complexity of the full algorithm.

\section{Assignment 4}
The maximum sum of a path in a pyramid is dependent on the maximum sum of the paths in the two sub-pyramids.
The maximum sum of these paths equal to the maximum sum of their sub-pyramids.
Repeating this pattern will result in finding the maximum sum path of a pyramid
of size 1, which is trivial. With this in mind, working from the bottom of the
pyramid determine the maximum of all subpyramids of the second to last row.
Repeating this will result in a pyramid of size one, with the value of the
maximum sum path. For $triangle.txt$ the maximum sum path will result in 7273.
\begin{listing}
\inputminted{python}{4.py}
\caption{Maximum sum path}
\end{listing}



\section{Assignment 5}
For the polynominal $p(i)=4i^{3} + 2i^{2} + 1$ there exists a function: $$q(n) =
an^{4} + bn^{3} + cn^{2} + dn + e$$ $$q(n) = \sum\limits_{i=1}^n p(i)$$ This
will result in the following 5 equations:
$$q(0) = e = 1$$
$$q(1) = a + b + c + d + e = 8$$
$$q(2) = 16a + 8b + 4c + 2d + e = 49$$
$$q(3) = 81a + 27b + 9c + 3d + e = 176$$
$$q(4) = 256a + 64b + 16c + 4d + e = 465$$
Solving these equations for the 5 unknown factors results in:
$$ a = 1, b = \frac{8}{3}, c = 2, d = \frac{4}{3}, e = 1$$
Substituting these variables back in $q(n)$ will result in the following
equation:
$$q(n) = n^{4} + \frac{8}{3} n^{3} + 2n^{2} + \frac{4}{3} n + 1$$
% =================================== REFERENCES ===================================

%\clearpage
%\bibliographystyle{unsrt}
%\bibliography{bib}

\end{document}
